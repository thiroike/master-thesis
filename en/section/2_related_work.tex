This chapter organizes related research concerning XX from various perspectives.

\subsection{Methods Based on XX}

Traditionally, various approaches have been taken to achieve XX.
In particular, the XX method~\cite{Furui2019-bz} deals with YY and can be expressed by the following equation:
%
\begin{align}
\hat{\theta} = \argmin_{\theta \in \mathcal{S}} \mathcal{J}(\theta).
\label{eq:conventional}
\end{align}
%
However, as shown in equation (\ref{eq:conventional}), this approach has limitations in terms of ZZ.
The proposed method resolves this issue by taking an approach based on AA.

\subsection{Methods Based on YY}

Meanwhile, approaches based on BB have also been proposed as methods to improve CC.
Among these, the XX method~\cite{Furui2021-ts} enables DD.

However, this method also has limitations due to EE and faces constraints in FF.
In contrast, the proposed method enables GG by implementing HH.