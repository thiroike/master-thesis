Fig.~\ref{fig:intro:sample} shows an overview of the proposed method.
The proposed method enables XX by utilizing YY.

%%%%%%%%%%%%%%%%%%%
% Figures and tables should be aligned to top or bottom. Ensure there is no text on both sides of figures/tables.
\begin{figure}[b] % For bottom alignment
\centering
\includegraphics[width=0.95\hsize]{figure/sample.pdf}
\caption{Sample figure.}
\label{fig:intro:sample}
\end{figure}
%%%%%%%%%%%%%%%%%%%

\subsection{Model Structure}

For a $D$-dimensional input vector $\mathbf{x} \in \mathbb{R}^D$, the structure of the proposed method is expressed by the following equation:
%
\begin{align}
p(\mathbf{x}) = \int p(\mathbf{x} | \boldsymbol{\mu}, u \boldsymbol{\Sigma}) p(u) \mathrm{d}u,
\end{align}
%
where $\boldsymbol{\mu} \in \mathbb{R}^D$ is the mean vector, $\boldsymbol{\Sigma} \in \mathbb{R}^{D \times D}$ is the covariance matrix, and $u \in \mathbb{R}^+$ represents the scale factor.

\subsection{Inference Method}

The proposed method enables inference through XX.

\subsection{Learning Algorithm}

The learning algorithm for the proposed method is as follows: