\documentclass[a4paper,platex,dvipdfmx,twocolumn]{jsarticle} 
\usepackage[left=20mm, right=20mm, top=20mm, bottom=20mm]{geometry}
\usepackage{graphicx}
\usepackage{tabularx, amssymb}
\usepackage{booktabs}
\usepackage{multirow}
\usepackage{afterpage}
\usepackage{titlesec}
\usepackage{cite}

\makeatletter
% --- \section の書式を jsarticle シングルカラムと同じ程度に ---
\titleformat{\section}{\reset@font\Large\gtfamily\bfseries}{\thesection}{1em}{}

% --- \subsection の書式を jsarticle シングルカラムと同じ程度に ---
\titleformat{\subsection}{\reset@font\large\gtfamily\bfseries}{\thesubsection}{1em}{}

% \titlespacing*{\section}{0pt}{2.3\baselineskip plus 0.5\baselineskip minus 0.2\baselineskip}{0.9\baselineskip}

% \titlespacing*{\subsection}{0pt}{1.5\baselineskip plus 0.4\baselineskip minus 0.2\baselineskip}{0.5\baselineskip}

% % タイトルと表のための独立したページ用のマクロを定義
% \makeatletter
\def\@maketitle{%
  \begin{center}
    {\LARGE 修 士 論 文 要 旨\par}
    \vspace{6pt}
    {Summary of the Master's Thesis\par}
    \vspace{20pt}
    % 表をタイトルページの一部として組み込む
    \begin{tabular}{|p{25mm}|p{50mm}|p{25mm}|p{50mm}|}
      \hline
      \centering{学 生 番 号} & M & \centering{プ ロ グ ラ ム} & 情報科学プログラム\\
      \centering{\scriptsize{Student ID number}} && \centering{\scriptsize{Program}} & \\
      \hline
      \centering{氏   名} & & \centering{主 指 導 教 員} & \\
      \centering{\scriptsize{Name}} && \centering{\scriptsize{Supervisor}} & \\
      \hline
      \centering{論 文 題 目} & \multicolumn{3}{|l|}{}\\
      \centering{\scriptsize{Thesis Title}} & \multicolumn{3}{|l|}{}\\
      \hline
    \end{tabular}
  \end{center}
  \vspace{30pt}  % タイトルと本文の間のスペース
}
% \makeatother

\begin{document}
% タイトルと表を表示
\twocolumn[{\maketitle}]

\section{はじめに}
これは広島大学大学院先進理工系科学研究科情報科学プログラムの修士論文要旨のテンプレートです。図表使用可,6ページ以内で記述してください。節の名前は適宜変更してください。

This is a template for a summary of the master's thesis in the Informatics and Data Science Program of the Graduate School of Advanced Science and Engineering, Hiroshima University. Figures and tables may be included; maximum length is 6 pages. Change the section names as appropriate.


\section{数式の記述}
数式の記述例を次に示します.


本研究では,前節での調査結果から,光源スペクトルの周波数成分に基づいて波長サンプル組数を決定する.
すなわち,光源スペクトルをフーリエ変換した高周波数成分を重み付け加算した値を用いて
波長サンプル組数$S$を,次式により決定する.
%
\begin{equation}
S = a\sum_{i=0}^{n}F(i)g(i)+c
\end{equation}
%
ここで,$a$は重み付け加算した高周波成分に対する寄与係数,
$c$は最小波長サンプリング組数,
$F(i)$は第$i$調波成分の値,$n$は最大高周波,
$g(i)$は高周波成分に対する重み関数であり,次式とする.
%
\begin{equation}
g(i) = {\left(\frac{i}{b}\right)}^\gamma
\end{equation}
%
ここで,$b$は重み付け関数の調整係数,$\gamma$は重み付け関数べき乗係数である.
本研究では,前節での調査結果に基づき$a=2.1$,$b=100$,$c=5$,$\gamma=0.25$に設定した.
\newpage

\section{図表について}
\subsection{図の挿入}
図の使用例を次に示します.
\newline

東広島キャンパスでは,図\ref{fig:campus}に示すように,
初夏にはアメリカ楓の美しい並木が緑に輝く.
%
\begin{figure}[t]
	\begin{center}
		\includegraphics[width=0.95\hsize]{figure/campus.jpg}
	\end{center}
	\vspace{-12pt} %図とキャプションとの間隔調整
	\caption{東広島キャンパスのアメリカ楓並木}
	\label{fig:campus}
\end{figure}
%
\newline

Minkowski Islandは4つのMinkowski Sausage (Quadratic type 2 curve) を
正方形状に配置したフラクタル図形である(図\ref{fig:fractal}参照).
%
\begin{figure}[b]
	\begin{center}
		\includegraphics[scale=1]{figure/fractal.pdf}
	\end{center}
	\vspace{-12pt} %図とキャプションとの間隔調整
	\caption{Minkowski Island}
	\label{fig:fractal}
\end{figure}
%
\newpage

\subsection{表の作成}
表の記入例を次に示します.

コースティックフォトンマップには,光源から放出されたフォトンが拡散反射面に到達する前に,
1回以上鏡面反射あるいは透過したフォトンの情報が格納される.
すなわち,表\ref{tab:path_notation}に示す表記法を用いれば,
{\tt LS+D}の経路をたどったフォトンの位置,放射束,入射方向が記録される.
%
\begin{table}[t]
  \begin{center}
  \caption{光の経路の表記法}
  \label{tab:path_notation}
  \begin{tabular}{l l | l l} \toprule
    {\tt L} & 光源 & {\tt (k)+} & {\tt k}が1回かそれ以上起きる \\ 
    {\tt E} & 視点 & {\tt (k)*} & {\tt k}が0回かそれ以上起きる \\
    {\tt S} & 鏡面反射 & {\tt (k)?} & {\tt k}が0回か1回起きる \\ 
    {\tt D} & 拡散反射 & {\tt (k|k')} & {\tt k}あるいは{\tt k'}が起きる \\ \bottomrule
  \end{tabular}
  \end{center}
\end{table}
%
\newline

\section{文献引用}
文献引用の例を次に示します.
\newline

レンダリング方程式(rendering equation)~\cite{kajiya1986rendering}は,算出すべき放射輝度$L$が式の両辺に表れた積分方程式となっている.

フォトンマッピング法~\cite{jensen2001realistic}では2段階のレイトレーシングを行うことによりレンダリング方程式を解く.


\section{まとめ}
卒業論文発表予稿の様式を示しました.


\bibliographystyle{bibstyle}
\bibliography{bib}

\end{document}