
本研究を遂行するにあたり,終始熱心なご指導と多大なるご鞭撻を賜りました,指導教員の古居彬准教授に心より感謝申し上げます.
古居准教授には,研究の方向性に関する議論から,論文執筆,プレゼンテーション技術に至るまで,細部にわたり丁寧にご指導いただきました.
配属当初は期待と不安が入り混じっておりましたが,先生の時に親身で,時に厳格なご助言のおかげで,迷うことなく研究に邁進することができました.

副指導教員として中間発表でアドバイスしてくださった向谷博明教授,曽智准教授に厚く御礼申し上げます.
先生方から頂いた鋭いご指摘は,本研究の質を高め,議論を深める上で極めて重要な指針となりました.

また,共同研究において多角的な視点から貴重なご意見をいただきました相澤宏旭助教に深く感謝致します.
学部時代の講義で得た基礎知識が,最先端の研究課題といかに結びつくかをご教示いただき,学問の奥深さと面白さを改めて実感する得難い機会を与えてくださいました.

日々の研究生活において,知能生体情報学研究室の皆様の存在は,私にとって大きな心の支えでした.
全体ゼミでの活発な議論はもとより,研究の合間に食事を共にし,日々の他愛もない会話で笑い合った時間は,私にとってかけがえのない思い出です.

最後に,私の大学院進学を快く認め,在学期間中,経済的・精神的に支え続けてくれた両親に,心からの感謝を捧げます.
何不自由なく研究に打ち込める恵まれた環境は,家族の献身的な支えがあったからこそ実現したものです.

ここに記して,深甚なる感謝の意を表します.