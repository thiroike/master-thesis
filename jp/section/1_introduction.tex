
医用画像セグメンテーションは,診断支援や治療計画において不可欠な技術であり,
正常組織または異常組織の領域を抽出することが求められる.
特に大腸ポリープ\cite{ji2022video}や頭頸部がん放射線治療における危険臓器\cite{maleki2020machine}の検出など,臨床応用が進んでいる.

しかしながら,医用画像セグメンテーションには固有の課題が存在する.特に重要な問題として,クラス不均衡が挙げられる.
医用画像では背景領域が大部分を占め,対象となる病変は相対的に小さい領域しか占めないことが多い.
この状況下では,従来の分類タスクで広く使われているCross-Entropy Loss\cite{long2015fully}は背景領域の学習に偏り,
臨床的に重要な小病変や曖昧な境界部分の正確なセグメンテーションが困難となる.

この課題に対処するため,クラス不均衡に対して頑健なDice Loss\cite{milletari2016v}やその
多くの拡張手法が提案され,CT画像\cite{zhu2019anatomynet, 9109297}や
MRI画像\cite{KATO2024107695}において高い性能が報告されている.
しかし,これらの損失関数は全画像に対して固定的な形状を持つという制約がある.医用画像の多様性を考慮すると,
画像ごとの難易度に応じて損失関数を適応的に調整することが望ましい.
このような適応的学習を実現するためには,損失関数の形状を制御する機構と,各画像の難易度を定量化する仕組みの2つが必要である.

本研究では,これら2つの要素を組み合わせた適応的学習フレームワークの構築を目指す.
形状制御には,Dice Lossを多項式展開して得られるPolyDice Loss\cite{polydice}を用いる.
難易度の定量化には,Monte Carlo Dropout\cite{pmlr-v48-gal16}(以下,MC Dropout)による
不確実性推定を用いる.MC Dropoutは推論時にDropoutを有効にすることで,モデルの認識的不確実性を効率的に推定し,
モデルが各画像をどの程度「難しい」と感じているかを定量化できる.
その後,得られた不確実性指標に基づき,PolyDice Lossの形状パラメータを動的に制御することで,
難しい画像には急峻な勾配を,簡単な画像には緩やかな勾配を与える適応的学習が可能である.