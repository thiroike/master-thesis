
本原稿は,広島大学大学院先進理工系科学研究科情報科学プログラムの修士論文での使用を想定した \LaTeX フォーマットである.
章・節の構成は適当なので,各自の内容に合わせて変更せよ.
フォントサイズや行間などをいじるとレイアウトが崩れる恐れがあるので,変更しないこと.

文中で従来文献を引用する際は,\verb|\cite|コマンドを使う.
このとき,reference listに記載してある論文のcitation keyを用いて,
「\verb|…に基づく方法が提案されている~\cite{Furui2019-bz}.|」
のように書く(\verb|~|によって引用記号だけが次の行に送られるのを防いでいる).

\verb|\label{ラベル名}| コマンドにより図や数式にラベルを付けることで,文中で \verb|\ref{ラベル}| と書けば図番号や数式番号を容易に取得できる.
例えば,\verb|\label{fig:method}| とラベル付けされた図は「\verb|Fig.~\red{fig:method}に提案法を示す|」で引用できるし,
\verb|\label{eq:model}| とラベル付けされた数式は「\verb|(\ref{eq:model})式に示す通り…|」のように引用できる.
手打ちで図番号や数式番号を入力することは避けよう.

そのほか,各種 \LaTeX コマンドを用いたサンプルは次章以降に載せているので,適宜参考にされたい.
なお,図(イラストやグラフ)はPDFファイルで用意すること.図ファイルの挿入方法は3章に載せている.

以下,2章ではなんらかの手法および関連の深い各種既存手法について述べる.
3章では提案する〇〇について説明し,4章で実験の方法,および結果と考察について述べる.
そして,最後に5章でまとめと今後の課題を示す.