
医用画像セグメンテーションは,診断支援や治療計画において不可欠な技術であり,
正常組織または異常組織の領域を抽出することが求められる.
特に大腸ポリープ\cite{ji2022video}や頭頸部がん放射線治療における危険臓器\cite{maleki2020machine}の検出など,臨床応用が進んでいる.

しかしながら,医用画像セグメンテーションには固有の課題が存在する.特に重要な問題として,クラス不均衡が挙げられる.
医用画像では背景領域が大部分を占め,対象となる病変は相対的に小さい領域しか占めないことが多い.
この状況下では,従来の分類タスクで広く使われているCross-Entropy Loss\cite{long2015fully}は背景領域の学習に偏り,
臨床的に重要な小病変や曖昧な境界部分の正確なセグメンテーションが困難となる.

この課題に対処するため,クラス不均衡に対して頑健なDice Loss\cite{milletari2016v}やその
多くの拡張手法が提案され,CT画像\cite{zhu2019anatomynet, 9109297}や
MRI画像\cite{KATO2024107695}において高い性能が報告されている.
しかし,これらの損失関数は全画像に対して固定的な形状を持つという制約がある.
医用画像は,撮影条件や個体差に加え,病変サイズ・形状の著しいばらつき,組織間のコントラストの違い,
あるいは境界の不明瞭さなどといった大きな多様性を有しており,画像ごとにセグメンテーションの難易度が大きく異なる.
固定的な損失関数を用いると,容易な画像と困難な画像に対して同一の学習信号を与えるため,困難な画像に対する学習が不十分となる可能性がある.
したがって,画像ごとの難易度に応じて損失関数を適応的に調整するアプローチが有望である.
このような適応的学習の実現には,2つの要素が求められる.1つは,難易度の高い画像に対して学習信号を強化できるよう,
損失関数の形状を柔軟に制御する手法である.
もう1つは,各画像がモデルにとってどの程度困難であるかを学習中に評価するための難易度指標である.

本研究では,これら2つの要素を統合した適応的学習フレームワークを提案する.
損失関数の形状制御には,Dice Lossを多項式展開して得られるPolyDice Loss\cite{polydice}を
用いる.PolyDice Lossは画像毎に最適なパラメータを用いて
損失関数の形状を連続的に制御できるため,適応的学習に適している.
難易度の定量化には,Monte Carlo Dropout\cite{pmlr-v48-gal16}(以下,MC Dropout)による
不確実性推定を用いる.
MC Dropoutは,推論時にDropoutを有効にすることで,モデルの認識的不確実性を効率的に推定できる.
この不確実性は,モデルがその画像のセグメンテーションにおいてどの程度の確信を持てないかを反映しており,セグメンテーション難易度の指標として利用可能である.
提案手法では,推定された不確実性指標に基づきPolyDice Lossの形状パラメータを動的に制御し,
モデルが現時点で確信を持てない画像には急峻な勾配を,既に安定した予測が可能な画像には緩やかな勾配を与えることで,効率的かつ頑健な学習の実現を目指す.

本研究の主要な貢献は以下の通りである.
\begin{enumerate}
    \item \textbf{不確実性に基づく画像難易度の動的定量化手法の導入:}
    MC Dropoutを用いて推論時の認識的不確実性を推定し,これを画像単位の「学習難易度」として定量化する手法を提案する.
    これにより,医用画像の多様なセグメンテーション難易度を,モデル自身の確信度に基づいて客観的に評価することを可能にする.

    \item \textbf{難易度に応じた損失関数の適応的制御フレームワークの構築:}
    定量化された難易度指標に基づき,PolyDice Lossの形状パラメータを適応的に制御する学習フレームワークを構築した.
    提案法は,学習の進行に伴い変化する難易度に応じて損失関数の勾配を自動調整することで,困難な症例への学習の注力と,容易な症例による勾配支配の抑制を同時に実現する.

    \item \textbf{複数のデータセットを用いた有効性と汎用性の実証:}
    医用画像データセットを用いた比較実験により,提案手法の有効性を検証した.
    実験の結果,従来の形状が固定された損失関数と比較してセグメンテーション精度が向上することを示した.
\end{enumerate}