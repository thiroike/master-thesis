
本章では,何らかの観点で何らかに関する関連研究を整理する.

\subsection{何らかに基づく方法}

従来,何らかを実現するためになんらかのアプローチが取られてきた.
特に,何らか法~\cite{Furui2019-bz}は,何とかかんとかであり,次式で表される.

\begin{align}
    \hat{\theta} = \argmin_{\theta \in \mathcal{S}} \mathcal{J}(\theta)
    \label{eq:conventional}
\end{align}

しかしながら,(\ref{eq:conventional})式から分かるように,何とかかんとかの面でこのアプローチには問題がある.
提案法はなんらかに基づくアプローチを取ることで,この問題を解決している.

\subsection{何とかに基づく方法}

一方,何らかかんとかを改善した方法として何らかに基づくアプローチも提案されてきた.
なかでも,何らか法~\cite{Furui2021-ts}はなんとかかんとかが可能である.

しかしながら,この方法にも何とかに起因する問題が存在しており,何とかに限界があった.
これに対し提案法では,何とかかんとかすることで,何らかが可能である.

