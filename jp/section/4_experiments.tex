提案法の有効性を検証するために,医用画像データセットを用いた実験を行った.


\subsection{実験設定}
\subsubsection{データセットおよび実装の詳細}
CVC-ClinicDBデータセット\cite{BERNAL201599}データセットおよびKvasir-SEGデータセット\cite{jha2020kvasir}を用いて実験を行った.
CVC-ClinicDBデータセットは$612$枚の大腸内視鏡画像($384 \times 288$ pixel),Kvasir-SEGデータセットは$1000$枚の大腸内視鏡画像で,
いずれも大腸内視鏡画像とそれに対応するポリープの正解マスクから構成される.

セグメンテーションモデルには表\ref{tab:unet_architecture}に示される構造のU-Net\cite{ronneberger2015u}を採用した.
学習にはAdam optimizer\cite{kingma2014adam}を使用し,バッチサイズ$32$, 学習率$10 ^ {-3}$に設定した.
損失関数にはPolyDice-1 Loss($\epsilon = 0$)を用いた.
前処理として全画像を$W = 224$ pixel, $H = 224$ pixelにリサイズし,訓練時には
$50\%$の確率で上下左右反転および明度・コントラストの変更を適用した.
最大エポック数$E$は$200$に設定した.

MC Dropoutによる不確実評価は,$\tau=10$エポックごとに実施した.
Dropout層はエンコーダの最終ブロックとデコーダの最終ブロックに配置し,
各評価時には,$p = 0.5$で$N = 10$回の確率的推論を行った.
またバッチ内の難易度指標の正規化時のパーセンタイルは$p = 25$に設定し,
適応的学習の開始エポックは$E_{\text{start}} = 0$とし,
$\epsilon_{\text{min}}=0$, $\epsilon_{\text{max}}=0.5$, $k=2$とした.


\subsubsection{比較条件および比較指標}
適応的学習の有効性を評価するため,複数の条件下で比較実験を行った.
まず,医用画像セグメンテーションにおける標準的なベースラインとしてDice Lossを検討し,
併せてクラス不均衡問題に対処する代表的手法であるFocal Loss\cite{lin2017focal}とも比較を行った.
また易しいサンプルの重みを下げる割合を調整するパラメータは$\gamma=2$と設定した.
次に,PolyDice-1 Lossにおけるパラメータ $\epsilon$を動的に制御することの有効性を検証するため,
$\epsilon$ を固定値とした場合との比較を行った.
これには,Dice Lossを近似した $\epsilon=0$ の設定に加え,予備実験により探索されたデータセットごとの最適な固定値を用いたPolyDice-1 Lossが含まれる.
これらの比較手法と,不確実性指標に基づいて $\epsilon$ を適応的に更新する提案法の
セグメンテーション精度を比較することで,提案手法の有効性を検証する.

またセグメンテーションの性能の評価には,Dice係数,IoU,Precision,Recallを用いた.

\clearpage

\begin{table}[t]
    \centering
    \caption{Overview of the U-Net Architecture}
    \label{tab:unet_architecture}
    \begin{tabular}{lc}
        \toprule
        \textbf{Layer} & \textbf{Output Size} \\
        \midrule
        % ★★★ 表示するテキストを追加 ★★★
        \multicolumn{2}{c}{\textit{--- Encoder ---}} \\
        Input & $224 \times 224 \times 3$ \\
        inc (DoubleConv) & $224 \times 224 \times 64$\\
        down1 (MaxPool + DoubleConv) & $112 \times 112 \times 128$ \\
        down2 (MaxPool + DoubleConv) & $56 \times 56 \times 256$ \\
        down3 (MaxPool + DoubleConv) & $28 \times 28 \times 512$ \\
        down4 (MaxPool + DoubleConv) & $14 \times 14 \times 512$ \\
        \midrule
        % ★★★ 表示するテキストを追加 ★★★
        \multicolumn{2}{c}{\textit{--- Decoder ---}} \\
        up1 (Upsample + DoubleConv) & $28 \times 28 \times 256$ \\
        up2 (Upsample + DoubleConv) & $56 \times 56 \times 128$ \\
        up3 (Upsample + DoubleConv) & $112 \times 112 \times 64$ \\
        up4 (Upsample + DoubleConv) & $224 \times 224 \times 64$ \\
        \midrule
        outc (Conv2d) & $224 \times 224 \times 2$ \\
        \bottomrule
    \end{tabular}
\end{table}

\clearpage

\begin{table}[t]
    \centering
    \caption{Performance Comparison with Existing Loss Functions on Kvasir-SEG Dataset}
    \label{tab:results}
    \begin{tabular}{ccccc} % lcccc から ccccc に変更(全て中央揃え)
        \toprule
        Method & Dice & IoU & Precision & Recall \\ % Dice Coefficient -> Dice に短縮
        \midrule
        Dice Loss & 0.7895 & 0.7021 & 0.8281 & 0.8154 \\
        Focal Loss ($\gamma = 2$) & 0.7192 & 0.6082 & 0.8634 & 0.6769 \\
        PolyDice-1 ($\epsilon=0$) & 0.8095 & 0.7198 & 0.8461 & 0.8268 \\ % fixed, .0 を削除
        PolyDice-1 (Opt. $\epsilon$) & 0.8095 & 0.7198 & 0.8461 & 0.8268 \\ % optimized, fixed を短縮
        \midrule
        Adaptive PolyDice-1 & \textbf{0.8272} & \textbf{0.7440} & \textbf{0.8707} & \textbf{0.8397} \\
    \bottomrule
    \end{tabular}
\end{table}