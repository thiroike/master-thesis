本稿では,医用画像セグメンテーションにおける適応的学習手法を提案した.
提案法では,MC Dropout を用いてモデルの認識的不確実性を定量化し,これを画像の
セグメンテーション難易度の指標として活用する.
得られた難易度指標に基づき,PolyDice-1 Loss の形状パラメータを動的に制御することで,
難しい画像には急峻な勾配を,簡単な画像には緩やかな勾配を与える適応的学習を実現した.

CVC-ClinicDB および Kvasir-SEG データセットを用いた評価実験において,
提案法は Dice Loss や FocalLoss などの既存手法に加え,
理想的な固定パラメータ設定をも上回る性能を達成した.
特に,従来手法では抽出が困難であった症例群に対して,CVC-ClinicDB において Dice 係数が$0.36$向上するなど,提案する適応的学習
戦略が困難な病変の検出に有効であることが示された.
今後はハイパーパラメータの自動最適化や,CT や MRIなどの 3 次元医用画像への拡張に取り組む予定である.