本稿では,医用画像セグメンテーションにおける適応的学習手法を提案した.
提案法では,MC Dropout を用いてモデルの認識的不確実性を定量化し,これを画像の
セグメンテーション難易度の指標として活用する.
得られた難易度指標に基づき,PolyDice-1 Loss の形状パラメータを動的に制御することで,
モデルの学習状態に応じた適応的な勾配調整を実現した.この動的制御により,結果として困難な症例の学習が促進されることを確認した.

CVC-ClinicDB および Kvasir-SEG データセットを用いた評価実験において,
提案法は Dice Loss や FocalLoss などの既存手法に加え,
テストデータに対して最適化された固定パラメータ設定をも上回る性能を達成した.
特に,従来手法では抽出が困難であった症例群に対して,CVC-ClinicDB において Dice 係数が$0.32$向上するなど,提案する適応的学習
戦略が困難な病変の検出に有効であることが示された.

本手法の限界として,不確実性推定のために複数回の推論を要するため,標準的な学習と比較して計算コストが増加する点が挙げられる.
また,現在の枠組みは経験的に設定されたハイパーパラメータに依存しており,これらをデータセットに応じて自動最適化する手法の確立が課題として残されている.
今後は,計算効率の改善に取り組むとともに,CTやMRIなどの3次元医用画像へ本手法を拡張し,より広範な臨床タスクにおける有効性を検証する予定である.