
Fig1に,提案法の概略図を示す.提案法は,信号計測部,前処理部,表面検出部,運動量評価部の4つの処理から構成される.
信号計測部で処理された頸動脈超音波動画像に対して前処理を施し,プラークの動きを抽出する.
その後,前処理後の動画像を表面検出部に入力し,運動評価部でJellyfish plaqueの定量評価を行う.

%%%%%%%%%%%%%%%%%%%
% 図表は上下に寄せる.図表の上下の両側に文がくることがないように注意.
\begin{figure}[b] % 下に寄せる場合
    \centering
    \includegraphics[width=0.95\hsize]{figure/sample.pdf}
    \caption{Sample figure.}
    \label{fig:intro:sample}
\end{figure}
%%%%%%%%%%%%%%%%%%%

\subsection{モデル構造}

$D$次元の入力ベクトル$\mathbf{x} \in \mathbb{R}^D$に対し,提案法の構造は以下の式で表される.
%
\begin{align}
    p(\mathbf{x}) = \int p(\mathbf{x} | \boldsymbol{\mu}, u \boldsymbol{\Sigma}) p(u) \mathrm{d}u
\end{align}
%
ここで,$\boldsymbol{\mu} \in \mathbb{R}^D$は平均ベクトル,$\boldsymbol{\Sigma} \in \mathbb{R}^{D \times D}$は共分散行列である.
また,$u \in \mathbb{R}^+$は尺度因子を表す.



\subsection{推論方法}

提案法は何らかの方法で推論が可能である.

\subsection{学習アルゴリズム}

提案法の学習アルゴリズムは以下の通りである.


